\section{Ejercicio 1}

\noindent En este trabajo se busca hallar, dados dos grafos simples $G_{1}$=($V_{1}$, $E_{1}$) y $G_{2}$=($V_{2}$, $E_{2}$), el máximo subgrafo común o MCS.\\
El máximo subgrafo común consiste en encontrar un grafo $H$=($V_{H}$, $E_{H}$) isomorfo tanto a un subgrafo de  $G_{1}$ como a un subgrafo de $G_{2}$ que maximice $\#E_{H}$.\\
El problema de grafos planteado en el paper es muy similar al de MCS (máximo subgrafo común) pero con la diferencia de que al plantearse un problema químico, el grafo representa una molécula, por lo que cada uno de sus nodos tiene un nombre asociado. En el caso del problema a resolver por Doc, los nodos son todos iguales (no tienen label). Para poder reutilizar algún algoritmo que resuelve el problema del paper, se puede tomar como parámetro de entrada el grafo en cuestión en el cual todos los nodos tienen la misma etiqueta.\\
En la química muchas veces es útil encontrar subesctructuras comunes entre moléculas, que se pueden encontrar a partir de representar una molécula con sus respectivos enlaces mediante un vértice por átomo y una arista por enlace, y aplicar algún algoritmo que resuelva MCS.\\
Aplicaciones en concreto pueden ser, por ejemplo, reducir la cantidad de experimentos necesarios para entender la capacidad de ciertas moléculas de actuar como drogas, ya que dado dos moléculas estructuralmente similares suelen tener ``actividad'' similar (suelen hacer efectos similares). Por lo que si uno quisiese estudiar un conjunto grande de moléculas para encontrar sus propiedades activas, en principio no sería necesario experimentar con todas, sino que podría agruparlas de a grupos con MCS lo suficientemente grandes y a partir de ahí estudiar las propiedades de cada grupo experimentando con unos pocos representantes por grupo.